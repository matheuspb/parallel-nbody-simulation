\documentclass[a4paper, 12pt]{article}

\usepackage[portuges]{babel}
\usepackage[utf8]{inputenc}
\usepackage{indentfirst}
\usepackage[urlcolor=blue, linkcolor=black, colorlinks=true]{hyperref}

\begin{document}
%%%%%%%%%%%%%%%%%%%%%%%%%%%%%%%%%%%%%%%%%%%%%%%%%%%%%%%%%%%%%%%%%%%%%%%%%%%%%%%%
\begin{titlepage}
    \begin{center}
        \huge{Universidade Federal de Santa Catarina}

\vspace{10pt}

        \vspace{85pt}

        \textbf{\LARGE{Trabalho 1}}
        \large{\\Simulação \textit{N-Body}}
        \vspace{160pt}

    \end{center}

    \begin{flushleft}
        \begin{tabbing}
            Alunos\qquad\qquad\=
            Matheus Silva Pinheiro Bittencourt\\
            \>Vinicius Macelai\\
            Professor\> Márcio Castro \\
        \end{tabbing}
    \end{flushleft}

    \begin{center}
        \vspace{\fill}
        Florianópolis, 16 de outubro de 2016
    \end{center}
\end{titlepage}

%%%%%%%%%%%%%%%%%%%%%%%%%%%%%%%%%%%%%%%%%%%%%%%%%%%%%%%%%%%%%%%%%%%%%%%%%%%%%%%%

\newpage
\tableofcontents
\thispagestyle{empty}

%%%%%%%%%%%%%%%%%%%%%%%%%%%%%%%%%%%%%%%%%%%%%%%%%%%%%%%%%%%%%%%%%%%%%%%%%%%%%%%%

\newpage
\pagenumbering{arabic}

\section{Descrição do problema}

O programa sequencial simula a interação gravitacional de um número "n" de
corpos em um espaço bi-dimensional. A posição e a massa das partículas são
geradas randomicamente. Cada passo da simulação é composto de basicamente duas
partes, a primeira, em que são calculadas as forças aplicadas individualmente em
cada partícula, e a segunda em que a nova posição de cada partícula é calculada,
baseado na força que é aplicada a ela. O número de partículas é o primeiro
parâmetro de entrada do programa.

A força aplicada em cada partícula é calculada individualmente, e para cada
partícula é calculado o quanto as outras partículas interagem com ela
gravitacionalmente, assim tem-se a força aplicada a cada partícula. Tendo as
forças aplicadas a cada partícula, é calculado a nova posição das partículas,
usando a velocidade que a partícula já possuia, mais a aceleração que a força
gravitacional gerou nela, já que $F = m*a$ . A cada passo da iteração é simulado
um $\Delta t$ diferente, que é calculado baseado na força máxima exercida em uma
partícula individual, dentre todas as partículas simuladas.

Esse processo de cálculo de forças e computação das novas posições das
partículas, é executado por um loop, um dado número de vezes, que é passado como
segundo parâmetro de entrada do programa. Para melhor comparação de desempenh
com a versão paralela adicionamos um terceiro parâmetro, que quando igual 
``0'', não imprime as posições finais das partículas na tela, se for qualque
outro valor, ele imprime.

%%%%%%%%%%%%%%%%%%%%%%%%%%%%%%%%%%%%%%%%%%%%%%%%%%%%%%%%%%%%%%%%%%%%%%%%%%%%%%%%

\section{Solução paralela}

Na solução paralela, procuramos paralelizar as duas rotinas principais do
programa sequencial, descritas na seção anterior, que são:
\begin{verbatim}
--> "nbody.c"
...
double ComputeForces(Particle[], Particle[], ParticleV[], int);
double ComputeNewPos(Particle[], ParticleV[], int, double);
...
\end{verbatim}

O primeiro passo, antes de paralelizar, foi colocar todas as variáveis que serão
compartilhadas entre as threads, como globais. Também colocamos como globais as
variáveis que já eram compartilhadas entre as duas rotinas, para evitar
passa-lás como parâmetro. As assinaturas das funções então ficaram:
\begin{verbatim}
--> "parallel.h"
...
void ComputeForces();
void ComputeNewPos();
...
\end{verbatim}

%%%%%%%%%%%%%%%%%%%%%%%%%%%%%%% < subsection > %%%%%%%%%%%%%%%%%%%%%%%%%%%%%%%%%

\subsection{Rotina ComputeForces()}

Para paralelizar a rotina \verb|ComputeForces()|, dividimos as iterações do
\verb|for| principal, e cada thread ficou responsável por uma `fatia' do loop.
A variável global \verb|chunk_size| armaneza o tamanho de cada fatia que vai ser
entregue a thread, e para cada thread é passado como parâmetro, a parte do
\verb|for| que será executado por ela. Por exemplo, simulando 20 partículas, com
4 threads, a variável \verb|chunk_size| será igual a 5 (\verb|npart/N_THREADS|),
e a thread 0 executará o loop de 0 até 4, a thread 1, de 5 até 9, a thread 2, de
10 até 14 e a thread 3, de 15 até 19, assim computando a força para todas as 20
partículas, igualmente distribuídas entre as threads. Isso é feito da seguinte
forma na rotina \verb|ComputeForcesThread| :
\begin{verbatim}
...
void* ComputeForcesThread(void* a) {
    int nt = *((int*)a);
    int begin = nt*chunk_size; // valor inicial de i
    int end = (nt+1)*chunk_size; // valor final de i

    for (int i = begin; i < end; ++i) {
    \* código semelhante a rotina não paralela *\
...
\end{verbatim}
Quando as threads são criadas, é passado como parâmetro o `número' da thread,
que vai de 0 até \verb|N_THREADS-1|, isso permite calcular aonde ela vai iniciar
o loop e aonde ele termina, como mostrado no código acima. Esse parâmetro é um
ponteiro para um inteiro que está no array global \verb|indexes|, esse array
possui seus valores iguais aos seus índices, ou seja, se \verb|N_THREADS = 5|,
então \verb|indexes = {0, 1, 2, 3, 4}|.
Agora, a rotina \verb|ComputeForces()| simplesmente cria as threads com
\verb|pthread_create(...)| e sincroniza elas com \verb|pthread_join(...)|.
Lembrando que a variável global \verb|max_f|, que antes era retorno dessa
rotina, pode ser modificada por múltiplas threads. Para evitar a condição de
corrida utilizamos um mutex:
\begin{verbatim}
...
pthread_mutex_lock(&max_lock);
if (fx > max_f) max_f = fx;
pthread_mutex_unlock(&max_lock);
...
\end{verbatim}

%%%%%%%%%%%%%%%%%%%%%%%%%%%%%%% < subsection > %%%%%%%%%%%%%%%%%%%%%%%%%%%%%%%%%

\subsection{Rotina ComputeNewPos()}

Nessa rotina foi feito um processo semelhante a rotina \verb|ComputeForces()|. O
for principal foi `fatiado' entre as threads igualmente. O processo feito para
definir qual thread executa qual parte do \verb|for|, é semelhante. Nesse caso
não há nenhuma condição de corrida, então não foi necessário uitilizar nenhum
mecanismo extra para sincronização de threads. No lugar do \verb|for| que
executava dentro da rotina sequencial, são criadas as threads, e depois elas são
sincronizadas (join).

%%%%%%%%%%%%%%%%%%%%%%%%%%%%%%%%%%%%%%%%%%%%%%%%%%%%%%%%%%%%%%%%%%%%%%%%%%%%%%%%

\section{Análise dos resultados}
\subsection{Plataforma utilizada}

\begin{tabbing}
    Processador:\qquad\qquad\=
    Intel® Core™ i3-4010U\\
    Frequência:\>1.7 GHz\\
    Cores(Threads):\>2(4)
\end{tabbing}

O número de ``Threads'' se deve a tecnologia da Intel descrita no site deles
por: ``Tecnologia Hyper-Threading Intel® (Tecnologia Intel® HT) oferece dois
segmentos de processamento por núcleo físico. Aplicativos altamente segmentados
podem fazer trabalhos adicionais paralelamente, concluindo as tarefas mais
rapidamente.''. As informações do processador foram retiradas do site:
\href{http://ark.intel.com/pt-br/products/75107/Intel-Core-i3-4010U-Processor-3M-Cache-1_70-GHz}{link}.

%%%%%%%%%%%%%%%%%%%%%%%%%%%%%%% < subsection > %%%%%%%%%%%%%%%%%%%%%%%%%%%%%%%%%

\subsection{Comparação de desempenho}

A seguir um tabela com o tempo dos resultados obtidos com diferentes números de
threads, e diferentes números de partículas (n\_part) e iterações (cnt), os
tempos foram obtidos fazendo a média de 3 execuções, usando o comando
\verb|time| do bash:
\begin{center}
    \begin{tabular}{ |c|c||c|c|c| }
    \hline
    n\_part & cnt & Sequencial & 2 Threads & 4 Threads \\
    \hline
    1000 & 100 & 3,122s & 1,576s & 1,408s \\
    \hline
    100 & 20000 & 5,189s & 3,561s & 4,39s \\
    \hline
    5000 & 10 & 7,186s & 3,861s & 3,503s \\
    \hline
    \end{tabular}
\end{center}
Podemos observar que o programa paralelo, no geral, é mais rápido, porém em
alguns casos é mais eficiente usar 2 threads, ao invés de 4, o que seria
natural, já que é o que o processador suporta. Podemos observar também que a
versão paralela tem maior vantagem quando computa um grande número de
partículas, e não tem tanto ganho de desempenho quando é um número de iterações
grande.

%%%%%%%%%%%%%%%%%%%%%%%%%%%%%%%%%%%%%%%%%%%%%%%%%%%%%%%%%%%%%%%%%%%%%%%%%%%%%%%%

\section{Observações sobre a execução}

O código possui um arquivo \verb|makefile|, e pode ser compilado usando o
comando \verb|make| do bash. Se compilado assim, serão gerados dois executáveis,
o \verb|nbody.out|, que é a versão sequencial, e o \verb|parallel.out|, que é a
versão paralela. Para executá-los, use o seguinte:
\begin{verbatim}
$ ./nbody.out arg1 arg2 arg3
$ ./parallel.out arg1 arg2 arg3 arg4
\end{verbatim}
Onde,
\begin{tabbing}
    arg1 = \=Número de partículas\\
    arg2 =\>Número de iterações\\
    arg3 =\>`0' para não printar os resultados, qualquer outro valor irá printar\\
    arg4 =\>Número de threads a serem usadas (caso seja a versão paralela)\\
\end{tabbing}

%%%%%%%%%%%%%%%%%%%%%%%%%%%%%%  < fim >  %%%%%%%%%%%%%%%%%%%%%%%%%%%%%%%%%%%%%%%

\end{document}

